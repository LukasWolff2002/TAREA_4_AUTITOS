\documentclass[letterpaper,12pt]{article}
\usepackage[bottom=2.5cm, top=2.5cm, right=2cm, left=3cm]{geometry}
\usepackage[spanish, es-tabla]{babel}
\usepackage{graphicx} 
\usepackage{hyperref}
\usepackage{booktabs}
\usepackage{natbib}
\usepackage{float}
\usepackage{listings}
\usepackage{xcolor}
\usepackage{parskip} 
\usepackage{fancyhdr} % Paquete para personalizar encabezados y pies de página
\usepackage{microtype}  % Mejora la justificación del texto

\hypersetup{
    colorlinks=true,
    linkcolor=black,
    citecolor=black,
    urlcolor=blue
}

% Configuración del encabezado
\pagestyle{fancy}
\fancyhf{} % Limpia los encabezados y pies de página actuales
\fancyhead[R]{\thepage} % Coloca el número de página en la parte superior derecha
\renewcommand{\headrulewidth}{0pt} % Elimina la línea horizontal en la parte superior de la página

\begin{document}

\begin{titlepage}
    \centering
    \vspace*{1cm}


    \textbf{\Large PARTICIÓN MODAL DE VIAJES DE VITACURA A LAS CONDES}
  
    \vspace{1cm}
    
    \textbf{Bernardo Caprile Canala-Echevarría, Felipe Alberto Vicencio Fossa y Lukas Wolff Casanova}\\
    Facultad de Ingeniería y Ciencias Aplicadas, Universidad de los Andes, Santiago de Chile\\
    e-mail: \href{mailto:bcaprile@miuandes.cl}{bcaprile@miuandes.cl}, \href{mailto:favicencio@miuandes.cl}{favicencio@miuandes.cl}, \href{mailto:lwolff@miuandes.cl}{lwolff@miuandes.cl}
    
    \vspace{2cm}
    
    \textbf{RESUMEN}
    
    \vspace{0.5cm}
    
    Texto del resumen que no exceda de 250 palabras. Este resumen proporciona una visión general concisa del contenido del trabajo, destacando los aspectos más relevantes y las principales conclusiones.

    \vspace{1cm}
    
    \textit{Palabras clave:} palabra1, palabra2, palabra3.
    
\end{titlepage}

\newpage

\section{Introducción}

La construcción de modelos para predecir el comportamiento de un sistema es algo común y necesario para la toma de decisiones, ya que permite anticiparse a posibles escenarios y planificar soluciones de manera más eficiente. En los sistemas de transporte, la necesidad de crear modelos es aún mayor, ya que estos permiten predecir cómo reaccionará la gente ante diferentes condiciones, como cambios en las rutas, tiempos de espera o precios. Gracias a estos modelos, es posible tomar decisiones informadas, optimizando los recursos y mejorando la calidad del servicio. 

Dentro de los modelos de transporte, uno de los más utilizados es el modelo de partición modal logit, que permite predecir la elección de un modo de transporte por parte de un individuo, basándose en sus preferencias y en las características de los modos disponibles. Este modelo es muy útil para predecir cómo se distribuirán los viajes entre diferentes modos de transporte, lo que es fundamental para la planificación de sistemas de transporte público y privado.

Para esta tarea, se generará un modelo simple de partición modal logit. Para ello, se ocupará la tabla de viajes basada en la encuesta Origen-Destino 2012 y se centrarán en los viajes que van desde Vitacura a Las Condes. Los parámetros que se tomarán en cuenta para la regresión será una variable binaria que indica si el viaje es en auto o no, la distancia entre el origen y el destino y el ingreso del pasajero o viajero.

\section{Resultados y Discusiones}

\subsection{Modelo Logit}

En primer lugar, el modelo logit utilizado tiene la siguiente estructura.

\begin{equation}
    V_a = \alpha_a + \theta_d X_d + \theta_i X_i,
\end{equation}

donde:
\begin{itemize}
    \item $\alpha_a$ es la constante modal del transporte privado,
    \item $\theta_d$ es el coeficiente de sensibilidad de la utilidad del auto respecto a la distancia,
    \item $X_d$ es la distancia de viaje,
    \item $\theta_i$ es el coeficiente de sensibilidad de la utilidad del auto respecto al ingreso,
    \item $X_i$ es el ingreso del viajero o pasajero.
    \item El valor obtenido para $V_a$ se compara a una utilidad sistemática definida como base arbitraria para la utilidad del transporte público, $V_b := 0$, de manera que como consecuencia lógica:
    \begin{equation}
        P_a = \frac{e^{V_a}}{1+e^{V_a}} \quad y \quad P_b = \frac{1}{1+e^{V_a}}.
    \end{equation} 
\end{itemize}
    
Luego, utilizando la librería \textit{scikit-learn} de \textit{Python}, se realizó la regresión lineal para obtener los parámetros del modelo logit.

\begin{table}[H]
    \centering
    \caption{Parámetros del modelo logit}
    \begin{tabular}{|c|c|}
    \hline
    Parámetro & Valor  \\ \hline
    $\alpha_{a}$    & 1.865     \\ \hline
    $\theta_{d}$    & -0.277     \\ \hline
    $\theta_{i}$    & 0.00135     \\ \hline
    $V_{a}$    & 1.612     \\ \hline
    \end{tabular}
    \label{tab:param}
\end{table}

\begin{table}[H]
    \centering
    \caption{Parámetros del Transporte Privado}
    \begin{tabular}{|c|c|}
    \hline
    Parámetro & Valor  \\ \hline
    $X_i$    & 606.8623     \\ \hline
    $X_{d}$    & 3.5522     \\ \hline
    \end{tabular}
    \label{tab:param}
\end{table}


Los valores anteriores obtuvieron una precisión del 89.47\% con las siguientes probabilidades:

\begin{table}[H]
    \centering
    \caption{Probabilidad de elección de transporte}
    \begin{tabular}{|c|c|}
    \hline
    Probabilidad & Valor  \\ \hline
    $P_{a}$    & 0.833    \\ \hline
    $P_{b}$    & 0.167     \\ \hline
    \end{tabular}
    \label{tab:prob}
\end{table}



\section{Análisis}

\subsection{Pregunta 1}

La partición modal observada en este par OD es un 83\% de personas viajan en transporte privado, mientras que el 17\% restante lo hace en transporte público. Esto se debe a que la utilidad del transporte privado respecto al ingreso es mayor ($\theta_{i}$), lo que se puede observar en la Tabla \ref{tab:param}.


\end{document}