\documentclass[letterpaper,12pt]{article}
\usepackage[bottom=2.5cm, top=2.5cm, right=2cm, left=3cm]{geometry}
\usepackage[spanish, es-tabla]{babel}
\usepackage{graphicx} 
\usepackage{hyperref}
\usepackage{booktabs}
\usepackage{natbib}
\usepackage{float}
\usepackage{listings}
\usepackage{xcolor}
\usepackage{parskip} 
\usepackage{fancyhdr} % Paquete para personalizar encabezados y pies de página
\usepackage{microtype}  % Mejora la justificación del texto

\hypersetup{
    colorlinks=true,
    linkcolor=black,
    citecolor=black,
    urlcolor=blue
}

% Configuración del encabezado
\pagestyle{fancy}
\fancyhf{} % Limpia los encabezados y pies de página actuales
\fancyhead[R]{\thepage} % Coloca el número de página en la parte superior derecha
\renewcommand{\headrulewidth}{0pt} % Elimina la línea horizontal en la parte superior de la página

\begin{document}

\begin{titlepage}
    \begin{center}
        
    
    \vspace*{1cm}


    \textbf{\Large PARTICIÓN MODAL DE VIAJES DE VITACURA A LAS CONDES}
  
    \vspace{1cm}
    
    \textbf{Bernardo Caprile Canala-Echevarría, Felipe Alberto Vicencio Fossa y Lukas Wolff Casanova}\\
    Facultad de Ingeniería y Ciencias Aplicadas, Universidad de los Andes, Santiago de Chile\\
    e-mail: \href{mailto:bcaprile@miuandes.cl}{bcaprile@miuandes.cl}, \href{mailto:favicencio@miuandes.cl}{favicencio@miuandes.cl}, \href{mailto:lwolff@miuandes.cl}{lwolff@miuandes.cl}
    
    \vspace{2cm}
    
    \textbf{RESUMEN}
    \end{center}
    \vspace{0.5cm}
    
    Para poder predecir el cómo reaccionará la gente ante diferentes condiciones, surge la necesidad de hacer modelos predictivos, para la tomar decisiones acertadas. Entre la gran variedad de modelos surge el modelo logit. Para este trabajo se ocupó para predecir la elección de transporte en viajes entre Vitacura y Las Condes, utilizando datos de la encuesta Origen-Destino 2012. El modelo considera la distancia del viaje y el ingreso del viajero como variables clave. Los resultados indican que la mayoría de las personas (83.3\%) eligen transporte privado, siendo el ingreso un factor positivo para su elección, mientras que la distancia actúa como un desincentivo. El modelo mostró una precisión del 89.47\%. Además, se analizaron posibles aumentos en los ingresos y su impacto en la elección de transporte, y se sugiere que incorporar variables adicionales, como la edad, podría mejorar la precisión del modelo, dando lugar a un modelo logit multinomial que permita una mejor representación de las preferencias de los viajeros.


    \vspace{1cm}
    
    \textit{Palabras clave:} Modelo logit, Vitacura, Las Condes, Partición modal, transporte
    
\end{titlepage}

\newpage

\section{Introducción}

La construcción de modelos para predecir el comportamiento de un sistema es algo común y necesario para la toma de decisiones, ya que permite anticiparse a posibles escenarios y planificar soluciones de manera más eficiente. En los sistemas de transporte, la necesidad de crear modelos es aún mayor, ya que estos permiten predecir cómo reaccionará la gente ante diferentes condiciones, como cambios en las rutas, tiempos de espera o precios. Gracias a estos modelos, es posible tomar decisiones informadas, optimizando los recursos y mejorando la calidad del servicio. 

Dentro de los modelos de transporte, uno de los más utilizados es el modelo de partición modal logit, que permite predecir la elección de un modo de transporte por parte de un individuo, basándose en sus preferencias y en las características de los modos disponibles. Este modelo es muy útil para predecir cómo se distribuirán los viajes entre diferentes modos de transporte, lo que es fundamental para la planificación de sistemas de transporte público y privado.

Para esta tarea, se generará un modelo simple de partición modal logit. Para ello, se ocupará la tabla de viajes basada en la encuesta Origen-Destino 2012 y se centrarán en los viajes que van desde Vitacura a Las Condes. Los parámetros que se tomarán en cuenta para la regresión será una variable binaria que indica si el viaje es en auto o no, la distancia entre el origen y el destino y el ingreso del pasajero o viajero.

\section{Resultados y Discusiones}

\subsection{Modelo Logit}

En primer lugar, el modelo logit utilizado tiene la siguiente estructura.

\begin{equation}
    V_a = \alpha_a + \theta_d X_d + \theta_i X_i,
\end{equation}

donde:
\begin{itemize}
    \item $\alpha_a$ es la constante modal del transporte privado,
    \item $\theta_d$ es el coeficiente de sensibilidad de la utilidad del auto respecto a la distancia,
    \item $X_d$ es la distancia de viaje,
    \item $\theta_i$ es el coeficiente de sensibilidad de la utilidad del auto respecto al ingreso,
    \item $X_i$ es el ingreso del viajero o pasajero.
    \item El valor obtenido para $V_a$ se compara a una utilidad sistemática definida como base arbitraria para la utilidad del transporte público, $V_b := 0$, de manera que como consecuencia lógica:
    \begin{equation}
        P_a = \frac{e^{V_a}}{1+e^{V_a}} \quad y \quad P_b = \frac{1}{1+e^{V_a}}.
    \end{equation} 
\end{itemize}
    
Luego, utilizando la librería \textit{scikit-learn} de \textit{Python}, se realizó la regresión lineal para obtener los parámetros del modelo logit.

\begin{table}[H]
    \centering
    \caption{Parámetros del modelo logit}
    \begin{tabular}{|c|c|}
    \hline
    Parámetro & Valor  \\ \hline
    $\alpha_{a}$    & 1.865     \\ \hline
    $\theta_{d}$    & -0.277     \\ \hline
    $\theta_{i}$    & 0.00135     \\ \hline
    $V_{a}$    & 1.612     \\ \hline
    \end{tabular}
    \label{tab:param}
\end{table}

\begin{table}[H]
    \centering
    \caption{Parámetros del Transporte Privado}
    \begin{tabular}{|c|c|}
    \hline
    Parámetro & Valor  \\ \hline
    $X_i$    & 606.8623     \\ \hline
    $X_{d}$    & 3.5522     \\ \hline
    \end{tabular}
    \label{tab:param}
\end{table}

\begin{table}[H]
    \centering
    \caption{Probabilidad de elección de transporte}
    \begin{tabular}{|c|c|}
    \hline
    Probabilidad & Valor  \\ \hline
    $P_{a}$    & 0.833    \\ \hline
    $P_{b}$    & 0.167     \\ \hline
    \end{tabular}
    \label{tab:prob}
\end{table}

Los coeficientes obtenidos muestran que la utilidad del transporte privado disminuye a medida que aumenta la distancia del viaje ($\theta_d = -0.277$). Por otro lado, el coeficiente asociado al ingreso ($\theta_i = 0.00135$) indica que, aunque el impacto es pequeño, las personas con ingresos más altos tienden a preferir el transporte privado sobre el público.

Además, la probabilidad de elegir el transporte privado ($P_a = 0.833$) es mayor que la del transporte público ($P_b = 0.167$). Esto indica una preferencia por el uso del auto en las condiciones estudiadas, lo cual puede estar relacionado tanto con factores económicos como de conveniencia.

Por último, la constante modal positiva para el transporte privado ($\alpha_a = 1.865$) indica una inclinación inicial hacia el uso del auto. Esto podría deberse a percepciones de comodidad o status asociadas al transporte privado, teniendo relación con la utilidad aleatoria.

En términos de precisión, el modelo logit logró un 89.47\%, lo cual indica un buen ajuste para predecir las elecciones de transporte de los individuos bajo estas circunstancias.

\section{Análisis}

Los coeficientes obtenidos muestran que la utilidad del transporte privado disminuye a medida que aumenta la distancia del viaje ($\theta_d = -0.277$). Por otro lado, el coeficiente asociado al ingreso ($\theta_i = 0.00135$) indica que, aunque el impacto es pequeño, las personas con ingresos más altos tienden a preferir el transporte privado sobre el público.

Además, la probabilidad de elegir el transporte privado ($P_a = 0.833$) es mayor que la del transporte público ($P_b = 0.167$). Esto indica una preferencia por el uso del auto en las condiciones estudiadas, lo cual puede estar relacionado tanto con factores económicos como de conveniencia.

Por último, la constante modal positiva para el transporte privado ($\alpha_a = 1.865$) indica una inclinación inicial hacia el uso del auto. Esto podría deberse a percepciones de comodidad o status asociadas al transporte privado, teniendo relación con la utilidad aleatoria.

En términos de precisión, el modelo logit logró un 89.47\%, lo cual indica un buen ajuste para predecir las elecciones de transporte de los individuos bajo estas circunstancias.

\subsection{Preguntas}

La partición modal observada en este par OD es un 83.3\% de personas viajan en transporte privado, mientras que el 16.7\% restante lo hace en transporte público. Esto se debe a que la utilidad del transporte privado respecto al ingreso es mayor ($\theta_{i}$), lo que se puede observar en la Tabla \ref{tab:param}.

Que el valor de $\theta_{d}$ sea negativo implica que la utilidad del transporte privado respecto a la distancia disminuye, lo que se traduce en que a medida que la distancia de viaje aumenta, la probabilidad de elegir transporte privado disminuye, es decir, se castiga la distancia.
Por otro lado, el valor de $\theta_{i}$ es positivo, lo que implica que la utilidad del transporte privado respecto al ingreso aumenta, lo que se traduce en que a medida que el ingreso del viajero aumenta, la probabilidad de elegir transporte privado también aumenta.
\\ \\
El valor $\alpha$ es la constante modal del transporte privado, es decir, es el valor de la utilidad del transporte privado respecto a la utilidad del transporte público cuando la distancia y el ingreso son iguales a cero. En este caso, el valor de $\alpha$ es 1.865, lo que implica que la utilidad del transporte privado es mayor que la del transporte público (1).
\\ \\
Para que los signos de los coeficientes cambien, se espera por ejemplo que un viaje sea mas eficiente mientras mayor sea la distanica, o al contrario, se aplique un desincentivo al uso del transporte privado mientras mayor sea el ingreso del viajero.


Suponiendo que el ingreso sube un 10\% para todos los viajeros, significa que el $X_i$ incrementa a 667.548. De esta forma, $V_a$ resulta en 1.782, lo que implica que la probabilidad de elegir transporte privado aumenta a 85.5\%, mientras que la probabilidad de elegir transporte público disminuye a 14.5\%.
\\ \\
De forma contraria, si el ingreso aumenta en un 12\% para la mitad de menores ingresos, mientras que aumenta un 8\% para la mitad de mayores ingresos se obtiene un $X_i$ igual a 672.642. De esta forma, $V_a$ resulta en 1.789, lo que implica que la probabilidad de elegir transporte privado aumenta a 85.6\%, mientras que la probabilidad de elegir transporte público disminuye a 14.4\%.


Los resultados mencionados anteriromente si hacen sentido matematicamente ya que la variacion del ingreso promedio se espera que no varie en gran medida entre ambos casos. Ahora, segun los modos de transporte, esta teoria no hace gran sentido ya que se deberia esperar un menor aumento en la eleccion de modos para el segundo caso, ya que se espera que un 12\% de aumento no sea capaz de contrarestar la baja del 10\% al 8\%.


Una manera de hacer mas preciso el modelo logit es involucrar una tercera variable, por ejemplo, la edad de los viajeros, de esta forma, se podría obtener un modelo logit multinomial, que permita obtener una mayor precisión en la elección de transporte. O tambien hacer una toma de datos proporcional a las probabilidades de los distintos modos segun encuestas anteriores, de esta manera, se espera eliminar el sesgo.



\section{Conclusión}

En este trabajo se generó un modelo de partición modal logit a partir de los datos de la encuesta Origen-Destino 2012, para predecir la elección de transporte privado y público en viajes desde Vitacura a Las Condes, en base a datos de longitud de viaje e ingreso del viajero.  

Los resultados obtenidos indican que la distancia del viaje tiene un impacto negativo en la elección del transporte privado, ya que el coeficiente $\theta_d$ es -0.277, lo que sugiere que a medida que la distancia del viaje aumenta, la utilidad percibida del auto disminuye. Por otro lado, el coeficiente de ingreso $\theta_i$ es positivo, aunque de baja magnitud (0.00135), indicando una ligera preferencia por el transporte privado a medida que el ingreso del viajero aumenta.

Además, se calculó la precisión del modelo obteniendo un 89.47\%, lo que valida su capacidad predictiva. Además, se observó que la probabilidad de elegir el automóvil privado ($P_a$) es significativamente mayor que la del transporte público ($P_b$), con valores de 0.833 y 0.167, respectivamente. Esto sugiere que en viajes entre estas dos comunas, los viajeros tienden a preferir el uso del automóvil privado, posiblemente debido a factores como la comodidad o el nivel de ingreso de la población estudiada.

Se concluye, que el modelo logit es una herramienta útil para predecir la elección de transporte en viajes intercomunales, y que la incorporación de variables adicionales, como la edad, podría mejorar la precisión del modelo, permitiendo una mejor representación de las preferencias de los viajeros.

\end{document}