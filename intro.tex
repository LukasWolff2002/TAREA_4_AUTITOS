\section{Introducción}

La construcción de modelos para predecir el comportamiento de un sistema es algo común y necesario para la toma de decisiones, ya que permite anticiparse a posibles escenarios y planificar soluciones de manera más eficiente. En los sistemas de transporte, la necesidad de crear modelos es aún mayor, ya que estos permiten predecir cómo reaccionará la gente ante diferentes condiciones, como cambios en las rutas, tiempos de espera o precios. Gracias a estos modelos, es posible tomar decisiones informadas, optimizando los recursos y mejorando la calidad del servicio. 

Dentro de los modelos de transporte, uno de los más utilizados es el modelo de partición modal logit, que permite predecir la elección de un modo de transporte por parte de un individuo, basándose en sus preferencias y en las características de los modos disponibles. Este modelo es muy útil para predecir cómo se distribuirán los viajes entre diferentes modos de transporte, lo que es fundamental para la planificación de sistemas de transporte público y privado.

Para esta tarea, se generará un modelo simple de partición modal logit. Para ello, se ocupará la tabla de viajes basada en la encuesta Origen-Destino 2012 y se centrarán en los viajes que van desde Vitacura a Las Condes. Los parámetros que se tomarán en cuenta para la regresión será una variable binaria que indica si el viaje es en auto o no, la distancia entre el origen y el destino y el ingreso del pasajero o viajero.