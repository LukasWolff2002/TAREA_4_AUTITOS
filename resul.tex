\section{Resultados y Discusiones}

\subsection{Modelo Logit}

En primer lugar, el modelo logit utilizado tiene la siguiente estructura.

\begin{equation}
    V_a = \alpha_a + \theta_d X_d + \theta_i X_i,
\end{equation}

donde:
\begin{itemize}
    \item $\alpha_a$ es la constante modal del transporte privado,
    \item $\theta_d$ es el coeficiente de sensibilidad de la utilidad del auto respecto a la distancia,
    \item $X_d$ es la distancia de viaje,
    \item $\theta_i$ es el coeficiente de sensibilidad de la utilidad del auto respecto al ingreso,
    \item $X_i$ es el ingreso del viajero o pasajero.
    \item El valor obtenido para $V_a$ se compara a una utilidad sistemática definida como base arbitraria para la utilidad del transporte público, $V_b := 0$, de manera que como consecuencia lógica:
    \begin{equation}
        P_a = \frac{e^{V_a}}{1+e^{V_a}} \quad y \quad P_b = \frac{1}{1+e^{V_a}}.
    \end{equation} 
\end{itemize}
    
Luego, utilizando la librería \textit{scikit-learn} de \textit{Python}, se realizó la regresión lineal para obtener los parámetros del modelo logit.

\begin{table}[H]
    \centering
    \caption{Parámetros del modelo logit}
    \begin{tabular}{|c|c|}
    \hline
    Parámetro & Valor  \\ \hline
    $\alpha_{a}$    & 1.865     \\ \hline
    $\theta_{d}$    & -0.277     \\ \hline
    $\theta_{i}$    & 0.00135     \\ \hline
    $V_{a}$    & 1.612     \\ \hline
    \end{tabular}
    \label{tab:param}
\end{table}

\begin{table}[H]
    \centering
    \caption{Parámetros del Transporte Privado}
    \begin{tabular}{|c|c|}
    \hline
    Parámetro & Valor  \\ \hline
    $X_i$    & 606.8623     \\ \hline
    $X_{d}$    & 3.5522     \\ \hline
    \end{tabular}
    \label{tab:param}
\end{table}

\begin{table}[H]
    \centering
    \caption{Probabilidad de elección de transporte}
    \begin{tabular}{|c|c|}
    \hline
    Probabilidad & Valor  \\ \hline
    $P_{a}$    & 0.833    \\ \hline
    $P_{b}$    & 0.167     \\ \hline
    \end{tabular}
    \label{tab:prob}
\end{table}

Los coeficientes obtenidos muestran que la utilidad del transporte privado disminuye a medida que aumenta la distancia del viaje ($\theta_d = -0.277$). Por otro lado, el coeficiente asociado al ingreso ($\theta_i = 0.00135$) indica que, aunque el impacto es pequeño, las personas con ingresos más altos tienden a preferir el transporte privado sobre el público.

Además, la probabilidad de elegir el transporte privado ($P_a = 0.833$) es mayor que la del transporte público ($P_b = 0.167$). Esto indica una preferencia por el uso del auto en las condiciones estudiadas, lo cual puede estar relacionado tanto con factores económicos como de conveniencia.

Por último, la constante modal positiva para el transporte privado ($\alpha_a = 1.865$) indica una inclinación inicial hacia el uso del auto. Esto podría deberse a percepciones de comodidad o status asociadas al transporte privado, teniendo relación con la utilidad aleatoria.

En términos de precisión, el modelo logit logró un 89.47\%, lo cual indica un buen ajuste para predecir las elecciones de transporte de los individuos bajo estas circunstancias.